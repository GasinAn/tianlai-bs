\documentclass[12pt]{ctexart}
\usepackage{amsmath}
\usepackage{amssymb}
\usepackage{amsthm}
\usepackage{graphicx}
\usepackage{hyperref}
\usepackage{syntonly}
\usepackage{textcomp}
\usepackage{verbatim}
%\syntaxonly
\hyphenation{}
\hypersetup{
	colorlinks,
	linkcolor=black,
	filecolor=black,
	urlcolor=black,
	citecolor=black,
}
\def\b{\boldsymbol}
\def\m{\mapsto}
\def\ra{\rightarrow}
\def\v{\verb}
\def\Ra{\Rightarrow}
\DeclareMathOperator{\sgn}{sgn}
\newtheorem{definition}{定义}
\newtheorem{theorem}{定理}
\title{天籁数据压缩研究}
\author{安嘉辰}
\begin{document}
\maketitle
本文探讨天籁数据的压缩, 参考文献为arXiv:1503.00638.
\tableofcontents
\section{基本步骤}
按参考文献中给出的方法, 数据压缩总共三步.
\begin{enumerate}
    \item 将数据(的实部和虚部)舍入成某个2的幂的整数倍,并保证此过程引入的不确定度小于观测数据本身的不确定度.
    \item 对数据执行\v|Bitshuffle|.
    \item 用\v|LZ4|(或\v|LZF|)对数据进行压缩.
\end{enumerate}
以下暂且将第1步称为bit round, 这一步是有损的, 另外两步是无损的.
\section{\texttt{Bitshuffle}+\texttt{LZ4}}
这两步可以直接用现有的轮子. 用Python/Cython进行压缩的话, 需要安装Cython, NumPy, h5py, hdf5plugin. 用Conda的话, 只需安装hdf5plugin, 其他的就都安装好了.

据h5py的文档, h5py的预构建版本是针对特定版本的HDF5构建的. 如果需要MPI支持或较新的HDF5功能, 则需要从源代码进行构建. 在Windows和Linux中从源代码构建h5py和hdf5plugin都是可行的. 因为从源代码构建时(和构建后)可能会出现各种奇奇怪怪的问题, 所以最好直接克隆源代码后用\v|python setup.py install|安装, 以便在出现问题时获取提示. 目前来看无论是预构建版本还是直接从源代码构建的版本都适用于天籁数据.

h5py内置了一个测试函数\v|run_tests|, 可以用此函数测试h5py是否正常安装. 此函数运行需要pytest和pytest-mpi轮子.

以下是进行\v|Bitshuffle|+\v|LZ4|压缩和解压的示例.
\begin{verbatim}
import h5py
import hdf5plugin
BS = hdf5plugin.Bitshuffle()
# Compress.
with h5py.File('hdf5.hdf5', 'w') as f:
f.create_dataset('data_name', data=data, **BS)
# Decompress.
with h5py.File('hdf5.hdf5', 'r') as f:
data = f['data_name'][...]
\end{verbatim}

\section{bit round}

\subsection{基本原理}

天籁Visbility观测数据为三维complex64数组, 其三个维度分别代表时间, 频率和``correlation product''. 这里的``correlation product''形如$(i,j)$, 其中$i$和$j$分别是两个``antenna channel''的编号. 数组第三维的索引是从0到某个整数(比如31)的, 每个索引都对应唯一一组``correlation product''. 天籁观测数据中有一个\v|blorder|数组记录了这个对应关系.

天籁Visbility观测数据的维度排列顺序和arXiv:1503.00638中讨论的观测数据不同. arXiv:1503.00638中讨论的观测数据将时轴作为变动最快的一轴. 文中称这样的排布有助于压缩, 但实际计算发现并不对压缩有明显帮助.

天籁Visbility观测数据满足自相关远远大于互相关的条件, 即$V_{ij}V_{ij}^{*}\ll V_{ii}V_{jj}, i\ne j$, 其中$i$和$j$是``antenna channel''的编号. 这样的话根据arXiv:1503.00638中给出的radiometer equation, 数据的理论不确定度$\sigma$满足
\begin{equation}
    \begin{cases}
        \sigma_{\text{Re},ii}^{2}=\frac{V_{ii}^2}{N},\\
        \sigma_{\text{Im},ii}^{2}=0,\\
        \sigma_{\text{Re},ij}^{2}=\frac{V_{ii}V_{jj}}{2N}, \; i\ne j,\\
        \sigma_{\text{Im},ij}^{2}=\frac{V_{ii}V_{jj}}{2N}, \; i\ne j.\\
    \end{cases}
\end{equation}
其中$N$为the number of samples entering the integrations, 对天籁数据而言为244140.

作者定义了一个协方差矩阵$C_{\alpha,ij;\beta,gh}$.其中$\alpha$,$\beta$是$\text{Re}$或$\text{Im}$, $C_{\alpha,ij;\beta,gh}$ 是$V_{ij}$的$\alpha$部和$V_{gh}$的$\beta$部的协方差.

这里作者定义的矩阵$C$, 其每一行或每一列对应一组$(\text{Re}/\text{Im}, i, j)$, 至于具体的对应方式随便.比如, 对应方式可能是行$1 \rightarrow (\text{Re}, 1, 1)$, 行$2 \rightarrow (\text{Im}, 1, 1)$, 行$3 \rightarrow (\text{Re}, 1, 2)$, 行$4 \rightarrow (\text{Im}, 1, 2)$\dots ,也可能是行$1 \rightarrow (\text{Re}, 1, 1)$, 行$2 \rightarrow (\text{Re}, 1, 2)$, 行$3 \rightarrow (\text{Re}, 2, 2)$, 行$4 \rightarrow (\text{Re}, 1, 3)$\dots . 无论如何都是不影响结论的. 在$V_{ij}V_{ij}^{*}\ll V_{ii}V_{jj}, i\ne j$时, 由radiometer equaition, $C$是对角阵, 且对角元是$\sigma_{\alpha,ij}^{2}$.

作者之后定义了一个行向量/列向量$s$: $s_{\alpha,ij}=\sqrt{1/(C^{-1})_{\alpha,ij;\alpha,ij}}$. 在$V_{ij}V_{ij}^{*}\ll V_{ii}V_{jj}, i\ne j$时, 由radiometer equaition, $C$是对角阵, 且对角元是$\sigma_{\alpha,ij}^{2}$, 此时可以证明$s_{\alpha,ij}^{2}=C_{\alpha,ij;\alpha,ij}=\sigma_{\alpha,ij}^{2}$.

对任意时刻, 任意频率, 任意$(i,j)$的观测数据的$\alpha$部$V_{\alpha,ij}$,按上面所说, 都可以算出一个$s_{\alpha,ij}$, 简记为$s$. 现在需要取一个控制bit round后数据精度的参量$f$, 这个$f$是bit round后the maximum fractional increase in noise,推荐值为$10^{-2}$到$10^{-5}$. 具体来说, 观测数据原有radiometer equaition中的不确定度$\sigma^2$, bit round引入了新的不确定度$\sigma_r^2$, 则保证$\sigma_r^2<f \sigma^2$.

随后按文中的推导, 需计算$g_{\text{max}}=\sqrt{12fs^2}$, 然后找到一个$g$, 使得:
\begin{itemize}
    \item $g \le g_{\text{max}}$\footnote{
        原文是$g < g_{\text{max}}$, 但$g = g_{\text{max}}$最多只会导致$\sigma_r^2=f \sigma^2$, 影响不大.},
    \item $g = 2^{b},\; b \in \mathbb{Z} $.
\end{itemize}

然后将观测数据$V_{\alpha,ij}$舍入成$g$的整数倍, 即求$n \in \mathbb{Z}$, 使得$n$与$V_{\alpha,ij}$同号, 且$ng-g/2\le\vert V_{\alpha,ij}\vert-\vert ng \vert< ng+g/2 $\footnote{
    原文还要求舍入是完全无偏的. 比如$g=1$时要将$0.5$舍入成$0$,~$1.5$舍入成$2$,~$2.5$舍入成$2$,~$4.5$舍入成$4$\dots~但舍入不完全无偏影响不大.
}.

比如, 若$V_{\alpha,ij}=1+1/2^2+1/2^3+1/2^4+1/2^7$, 则$g=1/2^4 \Rightarrow ng=1+1/2^2+1/2^3+1/2^4$, $g=1/2^3 \Rightarrow ng=1+1/2$, $g=1/2^2 \Rightarrow ng=1+1/2$, $g=1/2 \Rightarrow ng=1+1/2$, $g=1 \Rightarrow ng=1$, $g=2 \Rightarrow ng=2$, $g=4 \Rightarrow ng=0$.

将所有Visbility观测数据$V_{\alpha,ij}$都按上述方法舍入后, bit round完成.

\subsection{实现}

通用的实现在Github仓库GasinAn/h5bs中,为天籁数据专门设计的实现在Github仓库GasinAn/tianlai-bs中.

GasinAn/tianlai-bs中实现bit round的是dnb.pyx, 其中包含两个函数, 函数\v|reduce_precision|接受天籁观测数据的\v|vis|和\v|blorder|两个dataset(其中\v|vis|为Visbility观测数据)和$f/N$, 并在调用后对\v|vis|进行bit round. \v|reduce_precision|调用函数\v|bit_round|来完成bit round的最后舍入工作.

GasinAn/tianlai-bs中的test.py用于测试. 执行后test.py会要求输入天籁观测数据HDF5文件路径和参量$f$的值, 并返回压缩速度, 解压速度和压缩比.此程序在Windows和Linux中皆可正确运行.

\section{测试结果}
\begin{table}[!htb]
    \centering
    \begin{tabular}{c|cccc}
        \hline
        $f$&$10^{-2}$&$10^{-3}$&$10^{-4}$&$10^{-5}$\\
        \hline
        压缩速度(M/s)&$78$&$74$&$70$&$68$\\
        \hline
        解压速度(M/s)&$340$&$280$&$250$&$220$\\
        \hline
        压缩比&26.3\%&33.4\%&38.8\%&43.5\%\\
        \hline
    \end{tabular}
\end{table}
\section{进一步工作}
\begin{enumerate}
    \item 为提高效率, dnb.pyx中的函数\v|bit_round|使用了一种非常抽象的方法完成最后的数据舍入工作. 此函数经测试在普遍情形下是可以正确工作的, 然而其原理是否正确尚需考查.
    \item radiometer equation仅给出Visbility观测数据的理论不确定度, 实际上天籁Visbility观测数据的不确定度未必与radiometer equation给出的一致. 接下来需要将得到的压缩后的数据代入之前进一步处理数据的程序中, 看处理结果是否与未压缩的数据的处理结果相同.
    \item 程序运行时, 将观测数据完全读入内存, 经测试由于数据量太大, 内存占用过多, 程序运行速度会被大幅拖慢. 将尝试解决此问题, 提高压缩速度.
\end{enumerate}

\newpage
\pagestyle{empty}

\section*{附录~~函数$\!$\texttt{bit\_{}round}原理}

对任意$(v_0,\dots,v_{31})\in D:=\{0,1\}^{32}$,记$(v_0,\dots,v_{31})=(v_0\dots v_{31})$. 设映射$f\!:D\ra\mathbb{R},(v_0\dots v_{31})\m(1-2v_0)2^{(v_1\dots v_8)_2+127}(1.v_9\dots v_{31})_2$, $i\!:D\ra\mathbb{Z},(v_0\dots v_{31})\m(v_1\dots v_{31})_2-2^{32}v_0$, $u\!:D\ra\mathbb{N},(v_0\dots v_{31})\m(v_0\dots v_{31})_2$. 设映射$a\!:D^2\ra D,((v_0\dots v_{31}),(\tilde{v}_0\dots \tilde{v}_{31}))\m(\min\{v_0,\tilde{v}_0\}\dots\min\{v_{31},\tilde{v}_{31}\})$, $o\!:D^2\ra D,((v_0\dots v_{31}),(\tilde{v}_0\dots \tilde{v}_{31}))\m(\max\{v_0,\tilde{v}_0\}\dots\max\{v_{31},\tilde{v}_{31}\})$, $r\!:D\times\mathbb{N},((v_0\dots v_{31}),n)\m(v_0\dots v_0v_1\dots v_{31-n})$.

设映射$e\!:D\ra\mathbb{Z},(v_0\dots v_{31})\m(v_1\dots v_8)_2+127$, 则任意$v\in D$, $e(v)=i(r(a(v,c_e),23))+127$, 其中$c_e=(0111111110\dots0)$.

\begin{theorem}\label{t1}
    若$v=(v_0\dots v_{31})\in D$, $\tilde{v}=(\tilde{v}_0\dots \tilde{v}_{31})\in D$满足任意$j\in\{0\}\cup\{9,\dots,31\}$, $v_j=\tilde{v}_j=0$, 则$i(r(v,23))-i(r(\tilde{v},23))=i(r(i^{-1}(i(v)-i(\tilde{v})),23))$.
\end{theorem}
\begin{proof}
    若$i(v)\geqslant i(\tilde{v})$, 则$i(v)=2^{23}i(r(v,23))$, $i(\tilde{v})=2^{23}i(r(\tilde{v},23))$, $i(v)-i(\tilde{v})=2^{23}i(r(i^{-1}(i(v)-i(\tilde{v})),23))$, 定理\ref{t1}显然成立.

    若$i(v)< i(\tilde{v})$, 设$\mathrm{\delta}v=(\mathrm{\delta}v_0\dots \mathrm{\delta}v_{31})=i^{-1}(i(v)-i(\tilde{v}))$, 则$u(v)=i(v)=2^{23}(v_1\dots v_8)_2$, $u(\tilde{v})=i(\tilde{v})=2^{23}(\tilde{v}_1\dots \tilde{v}_8)_2$, $u(\mathrm{\delta}v)=i(\mathrm{\delta}v)+2^{32}=2^{23}(\mathrm{\delta}v_1\dots \mathrm{\delta}v_8)_2+2^{31}$, $u(r(v,23))=i(r(v,23))=(v_1\dots v_8)_2$, $u(r(\tilde{v},23))=i(r(\tilde{v},23))=(\tilde{v}_1\dots \tilde{v}_8)_2$, $u(r(\mathrm{\delta}v,23))=i(r(\mathrm{\delta}v,23))+2^{32}=(\mathrm{\delta}v_1\dots \mathrm{\delta}v_8)_2+2^8+\dots+2^{31}$, 所以$i(r(v,23))-i(r(\tilde{v},23))=(v_1\dots v_8)_2-(\tilde{v}_1\dots \tilde{v}_8)_2=(\mathrm{\delta}v_1\dots \mathrm{\delta}v_8)_2-2^8$, $i(r(i^{-1}(i(v)-i(\tilde{v})),23))=i(r(\mathrm{\delta}v,23))=(\mathrm{\delta}v_1\dots \mathrm{\delta}v_8)_2+2^8+\dots+2^{31}-2^{32}=(\mathrm{\delta}v_1\dots \mathrm{\delta}v_8)_2-2^8$, 故定理\ref{t1}成立.
\end{proof}

\begin{theorem}\label{t2}
    任意$v, \tilde{v}\in D$, $e(v)-e(\tilde{v})=i(r(i^{-1}(i(a(v,c_e))-i(a(\tilde{v},c_e))),23))$.
\end{theorem}
\begin{proof}
    由定理\ref{t1}易证.
\end{proof}

任意$g_{\text{max}}\in D$, 存在$g\in D$, 使得$\vert f(g)\vert=2^m(m\in\mathbb{Z})$, $f(g_{\text{max}})f(g)>0$, 且$\vert f(g)\vert\leqslant  \vert f(g_{\text{max}})\vert<2\vert f(g)\vert$, 设映射$T\!: D\ra D, g_{\text{max}}\m g$. 任意$v,g_{\text{max}}\in D$, 存在$n\in \mathbb{Z}$, 使得$nf(v)\geqslant 0$,且$(\vert n\vert-\frac{1}{2})\vert f(T(g_{\text{max}}))\vert\leqslant \vert f(v)\vert<(\vert n\vert+\frac{1}{2})\vert f(T(g_{\text{max}}))\vert$, 设映射$R\!:D^2\ra \mathbb{R}, (v,g_{\text{max}})\m n\vert f(T(g_{\text{max}}))\vert$.

\begin{theorem}
    任意$v,g_{\text{max}}\in D$, 若$f(g_{\text{max}})>0$, $e(v)\geqslant e(g_{\text{max}})$, 则$R(v,g_{\text{max}})=f(a(u^{-1}(u(v)+u(r(c_g,e(v)-e(g_{\text{max}})))),r(c_t,e(v)-e(g_{\text{max}}))))$, 其中$c_g=(00000000010\dots0)$, $c_t=(111111111110\dots0)$.
\end{theorem}
\begin{proof}
    若$\frac{f(T(g_{\text{max}}))}{2}<\vert2f(a(v,(1111111110\dots0)))-f(v)\vert$, 则$f(u^{-1}(u(v)+u(r(c_g,e(v)-e(g_{\text{max}})))))=f(v)+\sgn(f(v))\frac{f(T(g_{\text{max}}))}{2}$. 记$\tilde{v}=f^{-1}(f(v)+\sgn(f(v))\frac{f(T(g_{\text{max}}))}{2})$, 则存在$n\in \mathbb{Z}$, 使得$nf(\tilde{v})\geqslant0$,且$\vert n\vert f(T(g_{\text{max}}))\leqslant f(a(\tilde{v},r(c_t,e(v)-e(g_{\text{max}}))))<(\vert n\vert+1)f(T(g_{\text{max}}))$, 所以$f(a(\tilde{v},r(c_t,e(v)-e(g_{\text{max}}))))=R(v,g_{\text{max}})$.

    若$\frac{f(T(g_{\text{max}}))}{2}\geqslant\vert2f(a(v,(1111111110\dots0)))-f(v)\vert$, 则$R(v,g_{\text{max}})=2f(a(v,(1111111110\dots0)))$. 设$v=(v_0\dots v_{31})$, $\tilde{v}=(\tilde{v}_0\dots \tilde{v}_{31})=u^{-1}(u(v)+u(r(c_g,e(v)-e(g_{\text{max}}))))$,则$e(\tilde{v})=e(v)+1$, 则$2f(a(v,(1111111110\dots0)))=f((\tilde{v}_0\dots\tilde{v}_8 0\dots0))$, 并且任意$n\in\{9,\dots,9+e(v)-e(g_{\text{max}})-1\}$, $\tilde{v}_n=0$, 所以$f(a(\tilde{v},r(c_t,e(v)-e(g_{\text{max}}))))=f((\tilde{v}_0\dots\tilde{v}_8 0\dots0))=R(v,g_{\text{max}})$.
\end{proof}

\begin{theorem}
    任意$v=(v_0\dots v_{31})\in D$, $g_{\text{max}}=(g_{\text{max},0}\dots g_{\text{max},31})\in D$, 若$f(g_{\text{max}})>0$, $e(v)=e(g_{\text{max}})-1$, 则$R(v,g_{\text{max}})=f((v_0g_{\text{max},1}\dots g_{\text{max},8}0\dots0))$
\end{theorem}
\begin{proof}
    显然$\sgn(R(v,g_{\text{max}}))=\sgn(f((v_0g_{\text{max},1}\dots g_{\text{max},8}0\dots0)))$. $e(g_{\text{max}})=e(T(g_{\text{max}}))$, 则$e(v)=e(T(g_{\text{max}}))-1$, 则$\frac{f(T(g_{\text{max}}))}{2}\leqslant \vert f(v)\vert<f(T(g_{\text{max}}))$, 所以$\vert R(v,g_{\text{max}})\vert=f(T(g_{\text{max}}))=f((0g_{\text{max},1}\dots g_{\text{max},8}0\dots0))$.
\end{proof}

\begin{theorem}
    任意$v,g_{\text{max}}\in D$, 若$e(g_{\text{max}})-e(v)>1$, 则$R(v,g_{\text{max}})=0$.
\end{theorem}
\begin{proof}
    $e(g_{\text{max}})-e(v)>1$, 则$\vert f(v)\vert<\frac{\vert f(T(g_{\text{max}}))\vert}{2}$, 所以$R(v,g_{\text{max}})=0$.
\end{proof}
\end{document}
%\begin{verbatim}
%    
%\end{verbatim}
%对任意$v=(v_0\dots v_{31})\in D$, $i(r(v,23))=u(r(v,23))-2^{32}v_0$. 由定义,若%$v_0=0$, 则$u(r(v,23))=u(v)/2^{23}$, 所以$i(r(v,23))=u(v)/2^{23}-2^{32}v_0$, 所%以$i(r(a,23))-i(r(b,23))=u(a)/2^{23}-u(b)/2^{23}=(u(a)-u(b))/2^{23}$. 
%        
%$a_0=b_0=0$,则$i(a)=u(a),i(b)=u(b)$, 所以$i(r(i^{-1}(i(a)-i(b)),23))=i(r(i^{-1}%(u(a)-u(b)),23))$. 存在$p\in\mathbb{Z} $,使得$i(r(i^{-1}(u(a)-u(b)),23))=u(r(i^%{-1}(u(a)-u(b)),23))-2^{32}p=u(r(i^{-1}(u(a)-u(b)),23))-2^{32}p$.
%存在$p\in \mathbb{Z}$, 使得$u(a)-u(b)=i(a)-i(b)+2^{32}p$, 则$i^{-1}(i(a)-i(b))%=i^{-1}(u(a)-u(b)-2^{32}p)$. 另有$q\in \mathbb{Z}$, 使得$u(a)-u(b)-2^{32}p+2^%{32}q \in \{0,\dots,2^{32}-1\}$, 则$i^{-1}(u(a)-u(b)-2^{32}p)=u^{-1}(u(a)-u(b)%-2^{32}(p-q))$. 记$\tilde{d}=u(a)-u(b)-2^{32}(p-q)$,由定义$r(u^{-1}(\tilde{d}),%23)=u^{-1}(\tilde{d}/2^{23})$, 则有$i(r(i^{-1}(i(a)-i(b)),23))=i(u^{-1}(\tilde%{d}/2^{23}))$. 存在$n\in \mathbb{Z}$, 使得$i(u^{-1}(\tilde{d}/2^{23}))=\tilde{d}%/2^{23}+2^{32}n$
%
%\maketitle
%\tableofcontents
%\section{基本步骤及原理}
%把数据压缩总共三步.
%\begin{enumerate}
%    \item 对数据进行bit round.(有损)
%    \item 对数据进行bit shuffle.(无损)
%    \item 对数据用LZ4(或LZF)压缩.(无损)
%\end{enumerate}
%
%假设观测数据是按下面的方式存储的.
%\begin{verbatim}
%10101010101010101010100110101010 (续下行)
%10101010101010110010110110110111 (续下行)
%01001011011010110101110110110101 (续下行)
%10110110101101010101101010101010 (续下行)
%...
%\end{verbatim}
%
%首先把每个数据最后一些位舍入成0(即bit round), 比如像下面这样.
%\begin{verbatim}
%10101010101010101010000000000000 (续下行)
%10101010101000000000000000000000 (续下行)
%01001011011010000000000000000000 (续下行)
%10110110101101010000000000000000 (续下行)
%...
%\end{verbatim}
%
%然后进行bit shuffle., 完了以后像下面这样(不太确定\dots).
%\begin{verbatim}
%1101 ... (续下行)
%0010 ... (续下行)
%1101 ... (续下行)
%0001 ... (续下行)
%1110 ... (续下行)
%0001 ... (续下行)
%1111 ... (续下行)
%0010 ... (续下行)
%1101 ... (续下行)
%0010 ... (续下行)
%1111 ... (续下行)
%0001 ... (续下行)
%1010 ... (续下行)
%0001 ... (续下行)
%1000 ... (续下行)
%0001 ... (续下行)
%1000 ... (续下行)
%0000 ... (续下行)
%1000 ... (续下行)
%0000 ... (续下行)
%...
%0000 ... (续下行)
%0000 ...
%\end{verbatim}
%
%这样的话, 最后就有好多0, 然后就可以愉快地用LZ4等压缩了.
%\section{bit shuffle+LZ4}
%用Cython进行压缩,需要安装Python轮子: hdf5, h5py, hdf5plugin. 用conda即可.
%可能的问题: h5py与观测数据的HDF5版本不匹配. 若如此, 则需源码安装h5py, 我在Linux中好像%曾经能做到源码安装, Windows下没成功过. 并且, conda安装h5py后用其内置的测试函数来测%试, Linux没问题, Windows下会有点小错. 但无论如何, 现在conda安装的h5py是可以用在我从天%籁数据中抽出的样品上的.
%
%以下是进行bit shuffle+LZ4压缩和解压的示例.
%\begin{verbatim}
%import h5py
%import hdf5plugin
%BS = hdf5plugin.Bitshuffle()
%# Compress.
%with h5py.File('hdf5.hdf5', 'w') as f:
%f.create_dataset('data_name', data=data, **BS)
%# Decompress.
%with h5py.File('hdf5.hdf5', 'r') as f:
%data = f['data_name'][...]
%\end{verbatim}
%
%bit shuffle+LZ4已经很成熟了, 而且还用了很多高能操作, 我觉得很可靠, 也没有什么办法能改%进之.
%
%\section{bit round}
%这是核心部分.
%\subsection{数学原理}
%观测数据记录的是$V_{ij}$, 其中$i$, $j$是两个channel的编号, $V$是对应的Visibility. %在假设cross-correlation总是远比auto-correlation小,即$V_{ij}V_{ij}^{*}\ll V_{ii}V_%{jj}, i\ne j$时,有radiometer equaition如下.
%\begin{equation}
%    \begin{cases}
%        \sigma_{\text{Re},ii}^{2}=\frac{V_{ii}^2}{N}\\
%        \sigma_{\text{Im},ii}^{2}=0\\
%        \sigma_{\text{Re},ij}^{2}=\frac{V_{ii}V_{jj}}{2N}, \; i\ne j\\
%        \sigma_{\text{Im},ij}^{2}=\frac{V_{ii}V_{jj}}{2N}, \; i\ne j\\
%    \end{cases}
%\end{equation}
%其中$\sigma_{\text{Re}}^{2}$, $\sigma_{\text{Im}}^{2}$分别为Visibility实部和虚部%的不确定度, $N=\Delta_{\nu}\Delta_{t}$为the number of samples entering the %integrations(这玩楞我真不知是什么, 不知道对天籁数据这个值是多少).
%
%如果$V_{ij}V_{ij}^{*}\ll V_{ii}V_{jj}, i\ne j$不成立, 那么也有公式, 但接下来的计算%会需要不断给大大的矩阵求逆(对天籁数据是$528\times528$求$512\times3600$次,对我取的样%本而言),计算量暴增.据文章说对这种情况也能两眼一抹黑假装$V_{ij}V_{ij}^{*}\ll V_{ii}V_%{jj}, i\ne j$(不确定\dots).
%
%作者定义了一个协方差矩阵$C_{\alpha,ij;\beta,gh}$.其中$\alpha$,$\beta$是$\text{Re}$%或$\text{Im}$, $C_{\alpha,ij;\beta,gh}$ 是$V_{ij}$的$\alpha$部和$V_{gh}$的$\beta$%部的协方差.
%
%这里作者定义的矩阵$C$, 其每一行或每一列对应一组$(\text{Re}/\text{Im}, i, j)$, 至于具%体的对应方式随便.比如, 对应方式可能是行$1 \rightarrow (\text{Re}, 1, 1)$, 行$2 %\rightarrow (\text{Im}, 1, 1)$, 行$3 \rightarrow (\text{Re}, 1, 2)$, 行$4 %\rightarrow (\text{Im}, 1, 2)$\dots ,也可能是行$1 \rightarrow (\text{Re}, 1, 1)%$, 行$2 \rightarrow (\text{Re}, 1, 2)$, 行$3 \rightarrow (\text{Re}, 2, 2)$, 行%$4 \rightarrow (\text{Re}, 1, 3)$\dots . 无论如何都是不影响结论的. 在$V_{ij}V_{ij}%^{*}\ll V_{ii}V_{jj}, i\ne j$时, 由radiometer equaition, $C$是对角阵, 且对角元是%$\sigma_{\alpha,ij}^{2}$
%
%作者之后定义了一个行向量/列向量$s$: $s_{\alpha,ij}=\sqrt{1/(C^{-1})_{\alpha,ij;%\alpha,ij}}$. 在$V_{ij}V_{ij}^{*}\ll V_{ii}V_{jj}, i\ne j$时, 由radiometer %equaition, $C$是对角阵, 且对角元是$\sigma_{\alpha,ij}^{2}$, 此时可以证明$s_%{\alpha,ij}^{2}=C_{\alpha,ij;\alpha,ij}=\sigma_{\alpha,ij}^{2}$.
%
%对任意时刻, 任意频率, 任意$(i,j)$的观测数据的$\alpha$部$V_{\alpha,ij}$,按上面所说, %都可以算出一个$s_{\alpha,ij}$, 简记为$s$. 现在需要取一个控制bit round后数据精度的参%量$f$, 这个$f$是bit round后the maximum fractional increase in noise,推荐值为$10^%{-2}$到$10^{-5}$. 具体来说, 观测数据原有radiometer equaition中的$\sigma^2$, bit %round引入了新的$\sigma_r^2$, 则保证$\sigma_r^2<f \sigma^2$.为求精确, 可以在选择$f$%后再将$f$乘上$\min(1-V_{ij}V_{ij}^{*}/(V_{ii}V_{jj}))$或其他方法(没看懂另外的部分)%. 
%
%随后按文中的推导, 需计算$g_{\text{max}}=\sqrt{12fs^2}$, 然后找到一个$g$, 使得:
%\begin{itemize}
%    \item $g \le g_{\text{max}}$(原文是$<$, 但我觉得$\le$也成;$\le$应该最多只会导致%$\sigma_r^2=f \sigma^2$);
%    \item $g = 2^{b},\; b \in \mathbb{Z} $.
%\end{itemize}
%然后将观测数据$V_{\alpha,ij}$~bit round成$g$的整数倍, 即求$n \in \mathbb{Z}$, 使得%$n$与$V_{\alpha,ij}$同号, 且$ng-g/2\le\vert V_{\alpha,ij}\vert-\vert ng \vert< %ng+g/2 $(原文方法略有不同,之后说).
%
%比如, 若$V_{\alpha,ij}=1+1/2^2+1/2^3+1/2^4+1/2^7$, 则$g=1/2^4 \Rightarrow ng=1%+1/2^2+1/2^3+1/2^4$, $g=1/2^3 \Rightarrow ng=1+1/2$, $g=1/2^2 \Rightarrow ng=1%+1/2$, $g=1/2 \Rightarrow ng=1+1/2$, $g=1 \Rightarrow ng=1$, $g=2 \Rightarrow %ng=2$, $g=4 \Rightarrow ng=0$.
%
%把所有$V_{\alpha,ij}$~bit round成$ng$就完成第1步了.
%
%\section{float32\_{}bit\_{}round}
%设$D=\{0,1\}^{32}$, 则任意元素$v \in D$可以与一个实数对应. 设函数$f: %D\rightarrow\mathbb{R}, (a_0,\dots,a_{31})\mapsto (1-2a_0)2^{(a_1\dots a_8)%_2-127}(1.a_9\dots a_{31})_2$, 则$f$将$D$中任意元素映射成实数. 设函数%$u:D\rightarrow\mathbb{N}, (a_0,\dots,a_{31})\mapsto (a_0,\dots,a_{31})_2$, %$i:D\rightarrow\mathbb{Z}, v\mapsto u(v)-2^{32}a_0$, 则$u$和$i$将$D$中任意元素映%射成实数.
%
%设函数$s: D\rightarrow\mathbb{Z}, (a_0,\dots,a_{31})\mapsto(1-2a_0)$, $e: %D\rightarrow\mathbb{Z}, (a_0,\dots,a_{31})\mapsto(a_1\dots a_8)_2-127$, $m: %D\rightarrow\mathbb{R}, (a_0,\dots,a_{31})\mapsto(1.a_9\dots a_{31})_2$.
%
%设函数$r: D\times\mathbb{N}\rightarrow D, \left((a_0,\dots,a_{31}),n\right)%\mapsto(a_0,\dots,a_0,a_1,\dots,a_{31-n})$, $a: D^2\rightarrow D, \left((a_0,%\dots,a_{31}),(b_0,\dots,b_{31})\right)\mapsto(\min\{a_0,b_0\},\dots,\min\{a_%{31},b_{31}\})$, $o: D^2\rightarrow D, \left((a_0,\dots,a_{31}),(b_0,\dots,b_%{31})\right)\mapsto(\max\{a_0,b_0\},\dots,\max\{a_{31},b_{31}\})$.
%
%以下对任意$v=(v_0,\dots,v_{31})\in D$, 记$(v_0,\dots,v_{31})=(v_0\dots v_{31})$.%任取$v=(v_0\dots v_{31}),\tilde{g}=(\tilde{g}_0\dots \tilde{g}_{31})\in D$, 设%$d=e(v)-e(\tilde{g})$, $\tilde{e}_v=a(v,h_{\text{7F800000}})$, $\tilde{e}_%{\tilde{g}}=a(\tilde{g},h_{\text{7F800000}})$, 其中$h_{\text{7F800000}}=%(0111111110\dots0)$, 则$d=i(r(\tilde{e}_v,23))-i(r(\tilde{e}_{\tilde{g}},23))$.
%\begin{theorem}
%    对任意$a=(a_0\dots a_{31}),b=(b_0\dots b_{31})\in D$, 若任意$n\in \{0\}\cup \%{9,\dots,31\}$, $a_n=b_n=0$, 则$i(r(a,23))-i(r(b,23))=i(r(i^{-1}(i(a)-i(b)),%23))$.
%\end{theorem}
%\begin{proof}
%
%\end{proof}
